\documentclass[12pt, man]{apa6}
\usepackage[american]{babel}
\usepackage{csquotes}
\usepackage[style=apa,sortcites=true,sorting=nyt,backend=biber]{biblatex}
\DeclareLanguageMapping{american}{american-apa}
\addbibresource{ref.bib}



\begin{abstract}
    
\end{abstract}


\title{The Impact of Artificial Intelligence on Society}
\shorttitle{AI and Society}

\author{Garrett Price}
\affiliation{Brigham Young University}

\rightheader{AI and Society}
\begin{document}
\maketitle
Artificial Intelligence has been misunderstood by the masses ever since the term was conceived.  Media has greatly exaggerated the potential
benefits and drawbacks to this AI revolution.  These misconceptions of what artificial intelligence is and can do lead to concern. Many people question
whether AI has a place in our world and do not fully understand the benefits. Others are eager to implement automation in many aspects but are unaware of the
potential drawbacks.  For the purposes of this article, artificial intelligence refers to any form of automation by technology, such as cleaning robots or algorithms used on social media sites to display certain posts.\\
How has AI truly affected society and what needs to be done about it? This article seeks to address this question by combining the findings of studies from across the globe to clarify how AI can impact various aspects of life.  While the available research is extensive, it cannot provide a definitive answer of how AI will impact society due to the rapidly evolving nature of the field.  However, the results and conclusions from the current studies do provide enough information to see how AI has affected society up until now.  These effects can be changed according to how governments, tech firms, and employees react to the growing influence of AI.  This paper will address how AI has affected many aspects of life including consumer interactions, employment opportunities, and national economies.  It will discuss who benefits from automation and who is harmed by automation in each field.  The potential actions that can be taken to maximize or minimize the impact in each area will also be discussed.  First, the impact on individuals will be analyzed.  How AI can affect individuals in their consumer experiences and in their employment.  Next, the effects of artificial intelligence on society as a whole will be addressed.  These effects include economic changes and the spread of misinformation.  Lastly, the possible actions that governments, businesses, and individuals can make to limit or expand the effects of AI in the stated fields.
\newpage
\section*{Impact of AI on Individuals}
\subsection*{Individuals as consumers}
Consumers are having more interactions with AI as businesses automate more aspects of the consumer experience.  Any situation in which a consumer interacts with an AI, knowingly or not, to achieve something can be considered when discussing AI's impact on consumers.\\
\subsubsection*{Personalized Experiences}
The use of AI in consumer experiences leads to very different interactions with each consumer \parencite{Puntoni2021}.  These personalized experiences can make it easier for consumers to get what they want.  For example, one heavily personalized experience would be a social media feed.  The implemented algorithms collect data of the user and are able to find similar posts that the user is more likely to interact with.
This type of interaction with AI can be good for the consumer if it is what they expect.  Receiving recommendations based on interests can lead to a more fulfilling experience CITATION NEEDED??  However, this application of AI can lead to consumers feeling vulnerable and exploited if not handled properly \parencite{Aguirre2015}.

\subsubsection*{Invasion of Privacy}

\subsubsection*{Misunderstandings between AI and consumer}

\subsection*{Individuals as employees}
\subsubsection*{Unemployment}
\subsubsection*{Wages}
\subsubsection*{Reskilling and Upskilling}

\section*{Impact of AI on society}
\subsection*{Impact on Economy}

\subsection*{Impact on the masses}

\printbibliography

\end{document}